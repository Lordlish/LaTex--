\documentclass[UTF8, zihao=5, a4paper, oneside]{ctexart}
\usepackage{geometry}

\title{《声声慢》,沉郁中的暗流}
\author{{\kaishu 少科一班 \enspace 李书桁}}
\date{}
\geometry{scale=0.7}

\newcommand{\makenewtitle}[1]{\setlength{\parskip}{0.5em}\textbf{#1}\setlength{\parskip}{0em}}

\begin{document}
    \maketitle

    “愁”是历代文人墨客吟咏的一大主题,在宋词中也不例外。由于词人生活经历的多样性,即使同样是写愁情,不同词作所传达的具体情感、表达的艺术特色并不相同。有豪放的“愁”,如“酒酣胸胆尚开张,鬓微霜,又何妨?”(苏轼《江城子》);有凄婉的“愁”,如“惨愁颜、断魂无语。和泪眼、片时几番回顾,伤心脉脉谁诉”(柳永《鹊桥仙》);有细腻的“愁”,如“自在飞花轻似梦, 无边丝雨细如愁”(秦观《浣溪沙》)……词人笔下“愁”之丰富,可还是敌不过“千古第一才女”李清照的巅峰之作——《声声慢》。《声声慢》基调沉郁,但哀而不伤,其中丰富的创意如暗流般涌动,用巧夺天工的笔触创造出空前的佳作。

    \makenewtitle{一、叠词运用,折射复杂心理}

    词开篇即语出惊人,“寻寻觅觅,冷冷清清,凄凄惨惨戚戚”,七组叠词连用,这在宋词中十分罕见,甚至说是绝无仅有的。这种表达的妙处,一方面营造出冷淡凄清的氛围,奠定了全词忧郁的感情基调,浑然天成;另一方面在于音律和谐,增强了词作的音韵美,清代徐釚在《词苑丛谈》中谈到:“首句连下十四个叠字,真如大珠小珠落玉盘也。”词人“寻寻觅觅”,但只寻到“冷冷清清”,于是内心无比惆怅,“凄凄惨惨戚戚”。叠词的不断行进,就像女子轻盈的脚步走过,展现出词作清晰的脉络。

    不过,要想深入解读这句大名鼎鼎的词句,近停留在文字表面显然不够,正如孟子云:“颂其诗,读其书,不知其人,可乎?是以论其世也。”我们考察《声声慢》的创作背景,历来有两种不同的说法。一种认为这首词作于李清照中年时期,描写的是与赵明诚的相思之情;另一种观点则认为,创作时间是南渡以后,正值金兵入侵,北宋灭亡,丈夫去世,李清照心情沉痛,复杂的情感交织于心头,于是写下来这首《声声慢》,这也是目前最主流的看法。

    基于这样的背景,细究起来,李清照究竟“寻觅”的是什么呢?我认为有三点:其一是美满的爱情,李清照与赵明诚原本是相当恩爱的夫妻,在丈夫死后,这种幸福的状态被无情打破,让李清照拽入冰冷的深渊之中,让她心中充满了悲恸和对过去的怀念;其二是稳定的生活,金攻北宋,词人随朝廷流亡至浙东,饱尝颠沛流离之苦,也失去了大量财物,她内心深处一定苦苦寻找着当初那种安定无忧的生活;其三是振作的国家,北宋亡后,南宋政府偏安一隅,但这样的举措并没有换来长久的和平,金兵侵略行动依旧猖狂,给人民带来深重灾难,也让李清照等一众知识分子心怀忧虑,寻找救国救民之道。

    李清照所寻找的答案,不仅仅是自己的欢喜,更是国家的复兴、百姓的和乐,是整个民族的希望。但是她找到这个答案了吗?并没有。在“寻寻觅觅”紧接着就是“冷冷清清”,愿望破灭,“乍暖还寒”,词人又被残酷的现实笼罩。一扬一抑,一动一静,巨大的落差让读者体会到词人心中五味杂陈,复杂的心理交织,更显戏剧性。

    \makenewtitle{二、独特韵脚,展现才女本色}

    初读这首词时,其中对韵律独特的处理方式深深吸引了我,也给了我一些困惑。下阕“积”“损”“摘”“黑”等字读来似乎并不押韵,那是不是说它真的有问题呢?其实并不是这样。事实上,汉语言从古时候到现在已经发生了巨大的变化,从古代流传下来的有关韵脚、平仄的知识如今精通者不多,诗词的音韵美往往只能通过纸面来感受。例如,我们现在的普通话没有入声,“积”“摘”“黑”字很容易误判为平声,实际上它是入声,归于仄声一类。另外,许多字的古音与今音不同,例如“积”属于十三职部,“摘”属于十一陌部,“黑”属于十三职部,两部通用,尽管我们今天读起来可能感觉不到,但它们是可以押上的。

    当然,押韵只能算是诗人词人的基本功,李清照真正的实力表现在她大胆的韵脚选择。最令人印象深刻的显然是“黑”这个字了。 宋代张端义曾评价:“且秋词《声声慢》......更有一奇字云:‘守定窗儿,独自怎生得黑。’黑字不许第二人押。”(《贵耳集》卷上)“黑”字不仅新颖,是前人之未有过,而且对于词作的艺术表达起到了显著的作用。首先,“黑”反应了颜色的变化,由前文的“乍暖还寒”到“满地黄花堆积”,再到“独自怎生得黑”,色调逐渐黯淡,映照出词人内心情绪由希望到失望再到落寞的变化历程;其次,“黑”字还交代了时间的变化,词人从清晨开始发愁,一直守着窗儿直到傍晚,更加突显出词人内心的沉郁之情。

    我们关注到《声声慢》中的韵脚变化,这是其词作音韵之美的重要组成部分。它改变了原《声声慢》曲调的押平声韵的方式,转而押入声韵。这种变化使得词作的音律从原本的徐缓变得急促,从而更好地表达了词人内心的激荡和悲怆。李清照还巧妙地使用了叠字和双声字,这不仅增加了词句的音乐性,还强化了词人所要表达的情感。例如,词中的“到黄昏、点点滴滴”就运用了叠字的手法,使得词句在音韵上更具魅力,同时也更深刻地表达了词人的孤独和苦闷。可见,《声声慢》中的韵脚也展示了李清照对于语言的精湛驾驭能力。韵脚的变化与词句的内容、情感的表达紧密相连,共同构建了一个深邃而感人的艺术世界。

    \setlength{\parskip}{1em}
    在《声声慢》这首词中,李清照用她独特的艺术手法,让读者在感受她的孤独、苦闷之时,也能隐约感受到那份坚韧与不屈。她的词,就像一条在黑暗中奔涌的暗流,虽然深藏不露,但却具有强大的力量,能够引人深思,让人震撼。李清照的一生充满了坎坷和挫折,但她却从未放弃过对生活、对艺术的热爱和追求。她的词,就像她的人生一样,充满了变幻莫测的色彩。


\end{document}